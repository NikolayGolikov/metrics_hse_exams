\subsubsection{ИП, вспомнить всё!}

\begin{enumerate}
\item Найдите длины векторов $a=(1,1,1)$ и $b=(1,4,6)$ и косинус угла между ними. Найдите длину проекции вектора $b$ на вектор $a$.

\item Сформулируйте теорему Фалеса. Сформулируйте и докажите теорему Пифагора.

%\item На плоскости $\alpha$ лежит прямая $\ell$. Вне плоскости $\alpha$ лежит точка $C$. Ромео проецирует точку $C$ на прямую $\ell$ и получает точку $R$. Джульетта проецирует точку $C$ сначала на плоскость $\alpha$, а затем проецирует полученную точку $A$ на прямую $\ell$. После двух действий Джульетта получает точку $D$. Обязательно ли $R$ и $D$ совпадают?

\item Для матрицы

\[
A=\begin{pmatrix}
6 & 5 \\
5 & 6 \\
\end{pmatrix}
\]

\begin{enumerate}
\item Найдите собственные числа и собственные векторы матрицы
\item Найдите $\det (A)$, $\tr(A)$
\item Найдите собственные числа матрицы $A^{2017}$, $\det (A^{2017})$ и $\tr(A^{2017})$
\end{enumerate}

%\item Известно, что $X$ — матрица размера $n \times k$ и $n>k$, известно, что $X'X$ обратима. Рассмотрим матрицу $H=X(X'X)^{-1}X'$. Укажите размер матрицы $H$, найдите $H^{2016}$, $\tr(H)$, $\det(H)$, собственные числа матрицы $H$. Штрих означает транспонирование.

\item Занудная халява: известно, что $\Cov(X, Y)=5$, $\Var(X)=16$, $\Var(Y)=25$, $\E(X)=10$, $\E(Y)=-5$. Найдите $\Cov(X+2Y, Y-X)$, $\Var(X+2Y)$, $\E(X+2Y)$.

\item Блондинка Маша 100 раз выходила на улицу и при этом 40 раз встретила динозавра. Постройте 95\% доверительный интервал для вероятности встретить динозавра. На уровне 5\% проверьте гипотезу о том, что данная вероятность равна $0.5$ против альтернативной гипотезы об отличии данной вероятности от $0.5$.

\item В кошельке 5 монеток, три золотых и две серебряных. Маша берёт наугад две монетки по очереди. Маше достались одинаковые монетки. Какова условная вероятность того, что обе золотые?
\end{enumerate}





\subsection{Контрольная 1, 26.10.2017}

\begin{comment}

\input{kr1_test_bank.tex}


\subsubsection*{Часть 1. Тест.}

\onecopy{1}{

\cleargroup{test}
% \shufflegroup{2017_fall_retake_1}
\copygroup[10]{2017_fall_retake_1}{test}
\insertgroup{test}


%\AMCcleardoublepage
\clearpage

\AMCformBegin

% добавляем/убираем коммент
Ура! На этой страничке вопросов уже нет :)
%Это листок для ответов. Учитываются только ответы, перенесённые на этот листок.

\namefield{\fbox{
  \begin{minipage}{42em}
    Имя, фамилия и номер группы:\vspace*{3ex}\par
    \noindent\dotfill\vspace{2mm}
  \end{minipage}
}}


\vspace{2ex}

\AMCform


}

\end{comment}


\subsubsection*{Часть 2. Задачи.}

\begin{enumerate}

\item Докажите, что для модели парной регрессии  $Y_i = \beta_1 + \beta_2 X_i  + \e_i$, оцененной с помощью МНК, сумма остатков регрессии  $e_i=Y_i - \hat{Y}_i$ равна 0.

\item Покажите, что для регрессий $\hat{Y}_i= \hb_1 + \hb_2 X_i$ , $\hat{X}_i = \hat{\alpha}_1 + \hat{\alpha}_2 Y_i$, оценённых по одной и той же выборке $(X_1, Y_1),\ldots, (X_n, Y_n)$ коэффициенты детерминации $R^2$ совпадают, а оценки коэффициентов наклона связаны соотношением $\hb_2 \hat{\alpha}_2 = R^2$.

\item Для классической регрессионной модели $Y_i = \beta_1 + \beta_2 X_i + \e_i , i = 1,\ldots, 20$ известно, что $\sum_{i=1}^{20} X_i = 12, \sum_{i=1}^{20} X_i^2 = 12, \sum_{i=1}^{20} Y_i = 60, \sum_{i=1}^{20} Y_i^2 = 220, \sum_{i=1}^{20} X_i Y_i = 48$. Найдите
\begin{enumerate}
\item $\hb_1, \hb_2$, 
\item $TSS$, 
\item $R^2$, 
\item $\hs_{\e}^2$
\end{enumerate}

\item Для предыдущей задачи постройте точечный и 95\% интервальный индивидуальный прогноз в точке $X = 10$.

\item Заполните клетки с точками в приведенной ниже таблице. Клетки с XXX заполнять не надо.

\begin{tabular}{lr} \toprule
Показатель & Значение \\
\midrule
Multiple R          & XXX \\
$R^2$     			& \ldots \\
Standart error 		& XXX \\
Observations		& 50 \\
\bottomrule
\end{tabular}

ANOVA:

\begin{tabular}{lrrrrr} \toprule
      	   	 &  df 	& SS		& MS 	& F & Significance F \\
\midrule
Regression   & 1 	& 0.06  	& 	 	&  	& 		\\
Residual     & 48  	& 0.3   	&     	&  	&     	\\
Total        & 49  	& \ldots  	&    	&  	&     	\\
\bottomrule
\end{tabular}


\begin{tabular}{rrrrrr}
  \hline
 			& Coef. 	& St. error	& t-stat 	& Lower 95\% 	& Upper 95\% \\
  \hline
Intercept 	& 0.0044 	& 0.011 	& XXX 		& XXX 			& XXX \\
MARKET		& 0.52 		& 0.14 		& \ldots 	& \ldots 		& \ldots \\
   \hline
\end{tabular}


\end{enumerate}
