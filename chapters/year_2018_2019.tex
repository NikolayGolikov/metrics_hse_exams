\subsection{Вспомнить всё}

% нужен вариант, где  матрица 4  5
%                             5  4

\subsection{Вспомнить всё, ответы}

\begin{enumerate}
\item[2.]
\begin{enumerate}
  \item $\lambda^A_1 = -1$, $\lambda^A_2 = 9$,
  $h_1 = \begin{pmatrix}
  1 & -1
  \end{pmatrix}^T$,
  $h_2 = \begin{pmatrix}
  1 & 1
  \end{pmatrix}^T$
  \item $\det(A) = \lambda^A_1 \cdot \lambda^A_2 = -9$, $\tr(A) = \lambda^A_1 +
  \lambda^A_2 = 8$
  \item $\lambda^B_1 = 1 / \lambda^A_1 + 2018 = 2017$,
  $\lambda^B_2 = 1 / \lambda^A_2 + 2018 = 2018 + 1/9$,
  $\det(B) = \lambda^B_1 \cdot \lambda^B_2 = 2017 \cdot (2018 + 1/9)$,
  $\tr(B) = \lambda^B_1 + \lambda^B_2 = 4035 + 1/9$
\end{enumerate}
\item[3.]
\begin{enumerate}
\item $\left[20 - 1.96 \cdot 5 / \sqrt{100}; 20 + 1.96 \cdot 5 / \sqrt{100} \right]$
\item $z_{obs} = -4$, $z_{crit} = \pm 2.6$, основная гипотеза отвергается
\item $p-value \approx 0 $
\end{enumerate}
\item[4.] Заметим, что неважно, каким идёт Мефодий, а главное, что он самый сильный из
$22$ вышедших. Из $22$ вышедших всё равно есть кто-то самый сильный, и если его
считать Мефодием, то ничего не изменится. Значит, нам нужна вероятность того,
что второй лучший из всех попадёт на $22$ места из $33$, а самый лучший на $11$
мест из $32$ оставшихся. Искомая вероятность — $11/48$.

Или по формуле условной вероятности. Пусть $A$ означает, что Мефодий второй по силе
из всех, а $B$ — что он первый по силе из вышедших. Тогда
\[
\P(B) = 1/22,
\]
так как Мефодий должен быть самым сильным из вышедших, и
\[
\P(A \cap B) = 11/33 \cdot 1/32,
\]
так как на $22$-ом месте должен быть самый сильный из вышедших и второй по силе из
всех. Если он второй по силе из вышедших, то первый оказался среди $11$ невышедших,
и эта вероятность равна $11/33$, а вероятность быть вторым по силе среди всех равна
$1/32$. Итого,
\[
\P(A|B) = \frac{\P(A \cap B)}{\P(B)} = \frac{11}{48}.
\]
\end{enumerate}
