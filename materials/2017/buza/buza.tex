\documentclass[12pt]{article}

\usepackage{hyperref} % гиперссылки

\usepackage{tikz} % картинки в tikz
\usepackage{microtype} % свешивание пунктуации

\usepackage{array} % для столбцов фиксированной ширины

\usepackage{indentfirst} % отступ в первом параграфе

\usepackage{sectsty} % для центрирования названий частей
\allsectionsfont{\centering}

\usepackage{amsmath} % куча стандартных математических плюшек

\usepackage{comment} % добавление длинных комментариев

\usepackage[top=2cm, left=1.2cm, right=1.2cm, bottom=2cm]{geometry} % размер текста на странице

\usepackage{lastpage} % чтобы узнать номер последней страницы

\usepackage{enumitem} % дополнительные плюшки для списков
%  например \begin{enumerate}[resume] позволяет продолжить нумерацию в новом списке

\usepackage{caption} % что-то делает с подписями рисунков :)


\usepackage{fancyhdr} % весёлые колонтитулы
\pagestyle{fancy}
\lhead{Эконометрика}
\chead{}
\rhead{2018-06-04, БУЗА-VI}
\lfoot{}
\cfoot{}
\rfoot{\thepage/\pageref{LastPage}}
\renewcommand{\headrulewidth}{0.4pt}
\renewcommand{\footrulewidth}{0.4pt}



\usepackage{todonotes} % для вставки в документ заметок о том, что осталось сделать
% \todo{Здесь надо коэффициенты исправить}
% \missingfigure{Здесь будет Последний день Помпеи}
% \listoftodos — печатает все поставленные \todo'шки



\usepackage{booktabs} % красивые таблицы
% заповеди из докупентации:
% 1. Не используйте вертикальные линни
% 2. Не используйте двойные линии
% 3. Единицы измерения - в шапку таблицы
% 4. Не сокращайте .1 вместо 0.1
% 5. Повторяющееся значение повторяйте, а не говорите "то же"



\usepackage{fontspec} % что-то про шрифты?
\usepackage{polyglossia} % русификация xelatex

\setmainlanguage{russian}
\setotherlanguages{english}

% download "Linux Libertine" fonts:
% http://www.linuxlibertine.org/index.php?id=91&L=1
\setmainfont{Linux Libertine O} % or Helvetica, Arial, Cambria
% why do we need \newfontfamily:
% http://tex.stackexchange.com/questions/91507/
\newfontfamily{\cyrillicfonttt}{Linux Libertine O}

\AddEnumerateCounter{\asbuk}{\russian@alph}{щ} % для списков с русскими буквами
\setlist[enumerate, 2]{label=\asbuk*),ref=\asbuk*}

%% эконометрические сокращения
\DeclareMathOperator{\Cov}{Cov}
\DeclareMathOperator{\Corr}{Corr}
\DeclareMathOperator{\Var}{Var}
\DeclareMathOperator{\E}{E}
\def \hb{\hat{\beta}}
\def \hs{\hat{\sigma}}
\def \htheta{\hat{\theta}}
\def \s{\sigma}
\def \hy{\hat{y}}
\def \hY{\hat{Y}}
\def \v1{\vec{1}}
\def \e{\varepsilon}
\def \he{\hat{\e}}
\def \z{z}
\def \hVar{\widehat{\Var}}
\def \hCorr{\widehat{\Corr}}
\def \hCov{\widehat{\Cov}}
\def \cN{\mathcal{N}}



\usepackage[bibencoding = auto,
backend = biber,
sorting = none,
style=alphabetic]{biblatex}

\addbibresource{em1_pset_v2.bib}


\begin{document}

Минимум для подготовки к БУЗЕ, \cite{creel2018econometrics} :)


\begin{enumerate}

 \item Разное :)
 \begin{enumerate}
   \item Математический анализ
   \item Линейная алгебра
   \item Теория вероятностей
   \item Математическая статистика
 \end{enumerate}

 \item Метод Наименьших Квадратов до предположения о законе распределения $u$,
 \cite{schmidheiny2016guides}, \cite{decrouez2018mmdm}

\begin{enumerate}
 \item МНК-картинка;
 \item Нахождение всего-всего из $y$ и $X$;
 \item Матрица-шляпница;
 \item Ковариационная матрица как матрица-мать всех регрессий;
 \item Теорема Фриша-Вау или Во;
 \item LASSO, гребневая регрессия (ridge)
 \item {[не успели]} Квантильная регрессия
\end{enumerate}

 \item Теорема Гаусса-Маркова,  \cite{schmidheiny2016guides}
 \begin{enumerate}
 \item Формулировка и доказательство с детерминистическими регрессорами;
 \item Формулировки со стохастическими регрессорами
 \end{enumerate}

 \item МНК с нормальными ошибками,  \cite{schmidheiny2016guides}

 \begin{enumerate}
 \item Закон распределения оценок коэффициентов, $RSS$;
 \item Проверка гипотез об отдельном коэффициенте, о регрессии в целом, о системе линейных ограничений;
 \item Тест Чоу на стабильность коэффициентов;
 \item Тест Чоу на прогнозную силу;
 \end{enumerate}

 \item Метод максимального правдоподобия

 \begin{enumerate}
 \item Свойства оценок;
 \item Три теста (LM, Wald, LR);
 \item Функции правдоподобия стандартных моделей: линейная регрессия, логит/пробит, ARMA, ETS,
 линейная регрессия с гетероскедастичностью.
 \end{enumerate}

 \item Мультиколлинеарность;

 \item Методы снижения размерности:
 \begin{enumerate}
   \item Метод главных компонент, \cite{decrouez2016sm}
   \item {[не успели]} t-SNE;
 \end{enumerate}


 \item {[не успели]} Кластеризация;
 \begin{enumerate}
   \item Метод k-средних, \cite{decrouez2016sm}
   \item Иерархическая кластеризация
 \end{enumerate}


 \item Гетероскедастичность,  \cite{schmidheiny2016guides}
 \begin{enumerate}
 \item Определение, последствия;
 \item Тесты, графики;
 \item {[не успели]} Стьюдентизированные остатки;
 \item HC оценки ковариации;
 \item GLS и FGLS;
 \end{enumerate}

 \item Временные ряды, \cite{hyndman2014forecasting}, \cite{van2002time}.

 \begin{enumerate}
 \item Стационарный временной ряд, ACF, PACF;
 \item ARIMA-SARIMA;
 \item ETS;
 \item {[не успели]} тригонометрическое моделирование сезонности; \cite{pollock2010lectures}
 \item {[не успели]} преобразование Фурье; \cite{3blue1brown2017fourier}
 \item TBATS;
 \item DLM или модели состояние-наблюдение.
 \item {[не успели]} Фильтр Калмана, \cite{decrouez2016sm}
 \end{enumerate}


 \item Логит и пробит,  \cite{schmidheiny2016guides}
 \begin{enumerate}
 \item Описание моделей;
 \item Предельные эффекты;
 \item Чувствительность, специфичность;
 \item Кривая ROC — смотрим лекции :)
 \end{enumerate}

 \item Эндогенность,  \cite{schmidheiny2016guides}
 \begin{enumerate}
 \item Три примера: системы уравнений, пропущенные переменные, ошибки измерения;
 \item IV, двухшаговый МНК;
 \end{enumerate}

 \item GMM, \cite{creel2018econometrics}

 \item Модели панельных данных, \cite{schmidheiny2016guides}
 \begin{enumerate}
 \item  RE, FE, сквозная регрессии;
 \item  Тест Хаусмана;
 \end{enumerate}

 \item {[не успели]} Зелёные друзья, \cite{decrouez2018mmdm}
 \begin{enumerate}
 \item Классификационные деревья, случайный лес, градиентный бустинг;
 \end{enumerate}

 \item R. Разрешено всем пользоваться (заготовки, интернет), кроме общения.
 \begin{enumerate}
 \item Описательный анализ набора данных;
 \item Оценивание упомянутых выше моделей;
 \end{enumerate}

\end{enumerate}
­
\section*{Источники мудрости}
\printbibliography[heading=none]

\end{document}
