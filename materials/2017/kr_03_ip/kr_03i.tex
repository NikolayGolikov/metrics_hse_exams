\documentclass[12pt]{article}

\usepackage{tikz} % картинки в tikz
\usepackage{microtype} % свешивание пунктуации

\usepackage{array} % для столбцов фиксированной ширины

\usepackage{indentfirst} % отступ в первом параграфе

\usepackage{sectsty} % для центрирования названий частей
\allsectionsfont{\centering}

\usepackage{amsmath} % куча стандартных математических плюшек

\usepackage{comment}

\usepackage[top=2cm, left=1.2cm, right=1.2cm, bottom=2cm]{geometry} % размер текста на странице

\usepackage{lastpage} % чтобы узнать номер последней страницы

\usepackage{enumitem} % дополнительные плюшки для списков
%  например \begin{enumerate}[resume] позволяет продолжить нумерацию в новом списке
\usepackage{caption}


\usepackage{fancyhdr} % весёлые колонтитулы
\pagestyle{fancy}
\lhead{Эконометрика}
\chead{}
\rhead{2018-03-28, праздник номер три :)}
\lfoot{}
\cfoot{}
\rfoot{\thepage/\pageref{LastPage}}
\renewcommand{\headrulewidth}{0.4pt}
\renewcommand{\footrulewidth}{0.4pt}



\usepackage{todonotes} % для вставки в документ заметок о том, что осталось сделать
% \todo{Здесь надо коэффициенты исправить}
% \missingfigure{Здесь будет Последний день Помпеи}
% \listoftodos --- печатает все поставленные \todo'шки


% более красивые таблицы
\usepackage{booktabs}
% заповеди из докупентации:
% 1. Не используйте вертикальные линни
% 2. Не используйте двойные линии
% 3. Единицы измерения - в шапку таблицы
% 4. Не сокращайте .1 вместо 0.1
% 5. Повторяющееся значение повторяйте, а не говорите "то же"



\usepackage{fontspec}
\usepackage{polyglossia}

\setmainlanguage{russian}
\setotherlanguages{english}

% download "Linux Libertine" fonts:
% http://www.linuxlibertine.org/index.php?id=91&L=1
\setmainfont{Linux Libertine O} % or Helvetica, Arial, Cambria
% why do we need \newfontfamily:
% http://tex.stackexchange.com/questions/91507/
\newfontfamily{\cyrillicfonttt}{Linux Libertine O}

\AddEnumerateCounter{\asbuk}{\russian@alph}{щ} % для списков с русскими буквами
\setlist[enumerate, 2]{label=\asbuk*),ref=\asbuk*}

%% эконометрические сокращения
\DeclareMathOperator{\Cov}{Cov}
\DeclareMathOperator{\Corr}{Corr}
\DeclareMathOperator{\Var}{Var}
\DeclareMathOperator{\E}{E}
\def \hb{\hat{\beta}}
\def \hs{\hat{\sigma}}
\def \htheta{\hat{\theta}}
\def \s{\sigma}
\def \hy{\hat{y}}
\def \hY{\hat{Y}}
\def \v1{\vec{1}}
\def \e{\varepsilon}
\def \he{\hat{\e}}
\def \z{z}
\def \hVar{\widehat{\Var}}
\def \hCorr{\widehat{\Corr}}
\def \hCov{\widehat{\Cov}}
\def \cN{\mathcal{N}}


\begin{document}


Ровно 42 года назад польская путешественница Кристина Хойновская-Лискевич начала первое женское одиночное кругосветное плавание на парусной яхте. 
Плавание продлилось примерно два года. 

% На этой контрольной мы постараемся исследовать свойства ML оценок :)


\begin{enumerate}

  \item Рассмотрим задачу линейной регрессии для случая идеально точно известной дисперсии: $y=X\beta + u$, $u\sim \cN(0; I_{n\times n})$, 
    где $I_{n\times n}$ — единичная матрица. 
    Обозначим $s(\beta)$ — вектор-столбец, градиент логарифмической функции правдоподобия, 
    а $\hb$ — оценку параметров $\beta$ с помощью максимального правдоподобия.

    \begin{enumerate}
      \item Найдите $\Var(s(\beta)|X)$ и вспомните $\Var(\hat\beta|X)$;
      \item Упростите выражение $\Var(s(\beta)|X) \cdot \Var(\hat\beta|X)$;
    \end{enumerate}

  \item В созвездии Малой Медведицы водится $k$ видов медведепришельцев. 
    Исследователь Миша отлавливает $n$ медведепришельцев и классифицирует их по видам: 
    $y_1$ — количество медведепришельцев первого вида, $y_2$ — второго, \ldots, $y_k$ — $k$-го. 
    Миша хочет оценить вектор вероятностей $p=(p_1, \ldots, p_{k-1})$. 
    Вероятностей на одну меньше, чем видов, чтобы избежать жёсткой линейной зависимости между ними.
    \begin{enumerate}
      \item Как распределена в теории величина $y_1$? Чему равна её дисперсия?
      \item Как распределена в теории величина $y_{12}=(y_1 + y_2)$? Чему равна её дисперсия?
      \item Чему равна ковариация $y_1$ и $y_2$?
      \item Выпишите функцию правдоподобия с точностью до домножения на константу;
      \item Найдите $\hat p_{ML}$;
      \item Найдите $\Var(s(\theta))$ и $\Var(\hat p)$;
      \item Найдите предел $\lim \Var(s(\theta)) \cdot \Var(\hat p)$?
    \end{enumerate}

\end{enumerate}



\begin{enumerate}[resume]


  \item Исследовательница Несмеяна вывела хитрую формулу для $\hat a$ — несмещённой оценки неизвестного векторного параметра $a$.  
    Обозначим $s(a)$ — вектор-столбец градиент логарифмической функции правдоподобия. Докажите, что для оценки Несмеяны выполнено неравенство Крамера-Рао,
    а именно, матрица $M = \Var(s(a))\cdot \Var(\hat a) - I_{k\times k}$ положительна определена. 

     

Подсказки:

    \begin{enumerate}
      \item Вспомните, чему равно $\E(s(a))$. Достаточно просто вспомнить, доказывать не требуется!
      \item Найдите скаляры $\Cov\left(\hat a_1, \frac{\partial \ell}{\partial a_1}\right)$, 
	$\Cov\left(\hat a_1, \frac{\partial \ell}{\partial a_2}\right)$ 
	и матрицу $\Cov\left(\hat a, s(a) \right)$.
      \item Рассмотрим два произвольных случайных вектора $R$ и $S$ и два вектора констант подходящей длины $\alpha$ и $\beta$. 
	Найдите минимум функции $f(\alpha, \beta) = \Var(\alpha^T R + \beta^T S)$ по $\beta$. 
	Выпишите явно $\beta^*(\alpha)$ и $f^*(\alpha)$.
      \item Докажите, что для произвольных случайных векторов положительно определена матрица 
	\[
          \Var(R) - \Cov(R, S) \Var^{-1}(S)\Cov(S, R)
	\]
      \item Завершите доказательство векторного неравенства Крамера-Рао. 


    \end{enumerate}

  Без угрызений совести можно храбро переставлять интегралы и производные :)

  \newpage
  \item[3-лайт!] Утешительная версия задачи про Несмеяну. Если не получилось доказать векторную версию неравенства Крамера-Рао, то докажите скалярную :)

    Докажите, что для несмещённой скалярной оценки  $\Var(s(a))\cdot \Var(\hat a) \geq 1$.

    Подсказки:
	
    \begin{enumerate}
     
      \item Вспомните, чему равно $\E(\ell'(a))$. Достаточно просто вспомнить, доказывать не требуется!
      \item Найдите $\Cov(\hat a, \ell'(a))$;
      \item Сколько корней может быть у параболы $f(t)=\Var(R + tL)$? Каким может быть дискриминант параболы $f(t)$?
      \item  Докажите для произвольных случайных величин $R$ и $L$ неравенство Коши-Шварца, 
	\[
	  \Var(R) \cdot \Var(L) \geq \Cov^2(R, L) .
	\]
    \item Завершите доказательство скалярного неравенства Крамера-Рао :)
    \end{enumerate}


  \item[4.] Идея доказательства состоятельности ML оценки :)
    
    Пусть наблюдения $y_1$, \ldots, $y_n$ независимы и одинаково распределены с функцией плотности, зависящей от параметра $a$.
    Истинное значение параметра обозначим буквой $a_0$. Оценку максимального правдоподобия обозначим $\hat a$.

    Рассмотрим отмасштабированную логарифмическую функцию правдоподобия $\ell_n(a)=\ell(a) / n$, и
    ожидаемую логарифмическую функцию правдоподобия\footnote{Внимание: 
    ожидание считается с помощью истинного $a_0$ от функции, в которую входит константа $a$.}, 
    $\tilde \ell(a)=\E(\ell(a))$.
    \begin{enumerate}
      \item Что больше, $\ln x$ или $x-1$? Докажите!
      \item В какой точке находится максимум функции $\ell_n(a)$?        
      \item В какой точке находится максимум функции $\tilde \ell(a)$?
 
	Подсказка: рассмотрите выражение $\tilde \ell(a) - \tilde \ell(a_0)$ и примените доказанное неравество :)
      \item К чему сходится $\ell_n(a)$ по вероятности?
      %\item К чему сходится $\ell^{\prime\prime}_n(a)$? 

    \end{enumerate}


  \item[5.] Известна структура обратимой матрицы $M$,
    \[
         M = \begin{pmatrix}
	   A & B \\
	   0 & I_{k\times k} \\
	 \end{pmatrix}.
    \]

    \begin{enumerate}
      \item Найдите $M^{-1}$.
      \item Какие условия должны выполняться на блоки $A$ и $B$, чтобы $M$ была обратимой?
    \end{enumerate}


\end{enumerate}



\end{document}
