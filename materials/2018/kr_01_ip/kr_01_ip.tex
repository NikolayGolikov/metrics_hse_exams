\documentclass[12pt]{article} % размер шрифта

\usepackage{tikz} % картинки в tikz
\usepackage{microtype} % свешивание пунктуации

\usepackage{array} % для столбцов фиксированной ширины

\usepackage{url} % для вставки ссылок \url{...}

\usepackage{indentfirst} % отступ в первом параграфе

\usepackage{sectsty} % для центрирования названий частей
\allsectionsfont{\centering} % приказываем центрировать все sections

\usepackage{amsthm} % теоремы и доказательства

\theoremstyle{definition} % прямой шрифт в условии теорем
\newtheorem{theorem}{Теорема}[section]


\usepackage{amsmath, amssymb} % куча стандартных математических плюшек

\usepackage[top=2cm, left=1.5cm, right=1.5cm, bottom=2cm]{geometry} % размер текста на странице

\usepackage{lastpage} % чтобы узнать номер последней страницы

\usepackage{enumitem} % дополнительные плюшки для списков
%  например \begin{enumerate}[resume] позволяет продолжить нумерацию в новом списке
\usepackage{caption} % подписи к картинкам без плавающего окружения figure


\usepackage{fancyhdr} % весёлые колонтитулы
\pagestyle{fancy}
\lhead{Эконометрика, ИП}
\chead{}
\rhead{2018-10-26, Праздник-1}
\lfoot{}
\cfoot{}
\rfoot{\thepage/\pageref{LastPage}}
\renewcommand{\headrulewidth}{0.4pt}
\renewcommand{\footrulewidth}{0.4pt}



\usepackage{todonotes} % для вставки в документ заметок о том, что осталось сделать
% \todo{Здесь надо коэффициенты исправить}
% \missingfigure{Здесь будет картина Последний день Помпеи}
% команда \listoftodos — печатает все поставленные \todo'шки

\usepackage{booktabs} % красивые таблицы
% заповеди из документации:
% 1. Не используйте вертикальные линии
% 2. Не используйте двойные линии
% 3. Единицы измерения помещайте в шапку таблицы
% 4. Не сокращайте .1 вместо 0.1
% 5. Повторяющееся значение повторяйте, а не говорите "то же"

\usepackage{fontspec} % поддержка разных шрифтов
\usepackage{polyglossia} % поддержка разных языков

\setmainlanguage{russian}
\setotherlanguages{english}

\setmainfont{Linux Libertine O} % выбираем шрифт
% если Linux Libertine не установлен, то
% можно также попробовать Helvetica, Arial, Cambria и т.Д.

% чтобы использовать шрифт Linux Libertine на личном компе,
% его надо предварительно скачать по ссылке
% http://www.linuxlibertine.org/index.php?id=91&L=1

% на сервисах типа sharelatex.com этот шрифт есть :)

\newfontfamily{\cyrillicfonttt}{Linux Libertine O}
% пояснение зачем нужно шаманство с \newfontfamily
% http://tex.stackexchange.com/questions/91507/

\AddEnumerateCounter{\asbuk}{\russian@alph}{щ} % для списков с русскими буквами
\setlist[enumerate, 2]{label=\asbuk\cdot),ref=\asbuk\cdot} % списки уровня 2 будут буквами а) б) ...

%% эконометрические и вероятностные сокращения
\DeclareMathOperator{\Cov}{Cov}
\DeclareMathOperator{\Corr}{Corr}
\DeclareMathOperator{\Var}{Var}
\DeclareMathOperator{\E}{E}
\def \hb{\hat{\beta}}
\def \hs{\hat{\sigma}}
\def \htheta{\hat{\theta}}
\def \s{\sigma}
\def \hy{\hat{y}}
\def \hY{\hat{Y}}
\def \v1{\vec{1}}
\def \e{\varepsilon}
\def \he{\hat{\e}}
\def \z{z}
\def \hVar{\widehat{\Var}}
\def \hCorr{\widehat{\Corr}}
\def \hCov{\widehat{\Cov}}
\def \cN{\mathcal{N}}


\begin{document}

\begin{enumerate}
  \item Храбрый исследователь Вениамин поделил выборку на обучающую $(X, y)$ и тестовую $(X_{test}, y_{test})$.
  Регрессоры $X$ и $X_{test}$ Вениамин считает нестохастическими, а предпосылки
  теоремы Гаусса-Маркова — выполненными на всей исходной выборке. Естественно,
  $\hat y_{test} = X_{test}\hat\beta$, где $\hat\beta$ оценивается по обучающей выборке.

  Помогите Вениамину найти $\Var(\hat y_{test})$ и $\Cov(\hat \beta, \hat y_{test})$.

  \item Рассмотрим матрицу $X$ полного ранга с $n$ наблюдениями и $k$ столбцами.
  В каких границах могут лежать диагональные элементы матрицы-шляпницы $H$?
  Чему равно их среднее значение?

  Подсказка: найдите $\Var(\hat y)$ и $\Var(\hat u)$ в рамках предпосылок теоремы Гаусса-Маркова.

  \item Рассмотрим стандартный $t$-тест на равенство некоторого коэффициента бета нулю.
  Докажите, что
  \[
         t^2 = \frac{RSS_r - RSS_{ur}}{RSS_{ur}/(n-k)},
  \]
  где $RSS_r$ — сумма квадратов остатков в модели без тестируемого коэффициента
  (выкинут регрессор при проверямом коэффициенте),
  $RSS_{ur}$ — аналогичная сумма в модели с включённым тестируемым коэффициентом, $k$ —
  число оцениваемых коэффициентов бета в модели с тестируемым коэффициентом, $n$ —
  количество наблюдений.

  Утешительный приз: упростите эту формулу для случая парной регрессии и докажите её :)

  \item Рассмотрим стандартную ошибку оценки коэффициента бета при регрессоре $z$
  в множественной регрессии.
  Докажите, что

  \[
         se^2(\hat\beta_z) = \frac{RSS / (n-k)}{\sum (z_i - \bar z)^2} \cdot \frac{1}{1 - R^2_z},
  \]
  где $R^2_z$ — коэффициент детерминации во вспомогательной регресии объясняющей переменной
  $z$ на остальные объясняющие переменные.

  Утешительный приз: упростите эту формулу для случая парной регрессии и докажите её :)


   \item Винни-Пух нашёл случайный вектор $w$ и одномерную случайную величину $z$.
   Также он узнал, что $\Corr(w, w) = A$ и $\Corr(w, z) = b$.

   К сожалению, у Винни-Пуха опилки в голове, а он очень хочет найти такую линейную комбинацию
   компонент вектора $w$, которая была бы сильнее всего коррелирована со случайной
   величиной $z$.

   Помогите Винни-Пуху!

   Как выглядят веса этой линейной комбинации?
   Чему равна максимально возможная корреляция?

 \item Машенька построила парную регрессию по 11 наблюдениям с $R^2=
0.95$. Чтобы напакостить Машеньке, Вовочка переставил в случайном
порядке значения зависимой переменной и предложил Машеньке заново оценить модель.

Какой ожидаемый $R^2$ получит Машенька?


\end{enumerate}



\end{document}
