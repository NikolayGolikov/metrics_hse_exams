\element{2016_kr_04_final}{ % в фигурных скобках название группы вопросов
%  %\AMCnoCompleteMulti
\begin{questionmult}{1} % тип вопроса (questionmult — множественный выбор) и в фигурных — номер вопроса
С помощью МНК оценена зависимость потребления $Y_i$ от дохода $X_i$, $\hat Y_i = 0.5 - 0.3 X_i$. Если же использовать центрированные и нормированные переменные, то зависимость примет вид $\hat Y_i^{st} = -0.7 X_i^{st}$. Коэффициент множественной детерминации $R^2$ для первой модели равен

\begin{multicols}{3} % располагаем ответы в 3 колонки
\begin{choices} % опция [o] не рандомизирует порядок ответов
       \correctchoice{$0.49$}
       \wrongchoice{$0.09$}
       \wrongchoice{$0.3$}
       \wrongchoice{$0.7$}
       \wrongchoice{$0.21$}
    \end{choices}
   \end{multicols}
\end{questionmult}
}



\element{2016_kr_04_final}{ % в фигурных скобках название группы вопросов
%  %\AMCnoCompleteMulti
\begin{questionmult}{2} % тип вопроса (questionmult — множественный выбор) и в фигурных — номер вопроса
Оценка $\hb_{2SLS}$  модели  $Y = X\beta + \e$ получена двухшаговым МНК с матрицей инструментальных переменных  $Z$. Если число инструментов превышает количество включенных в модель факторов, то $\hb_{2SLS}$ имеет вид

\begin{multicols}{2} % располагаем ответы в 3 колонки
\begin{choices} % опция [o] не рандомизирует порядок ответов
       \correctchoice{$(X'Z(Z'Z)^{-1}Z'X)^{-1}X'Z(Z'Z)^{-1}Z'Y$}
       \wrongchoice{$(X'Z(Z'Z)^{-1}Z'X)^{-1}Z'Z(Z'Z)^{-1}X'Y$}
       \wrongchoice{$(Z'X)^{-1}Z'Y$}
       \wrongchoice{$(Z'Z)^{-1}Z'Y$}
       \wrongchoice{$Z(Z'Z)^{-1}Z'X$}
    \end{choices}
\end{multicols}
\end{questionmult}
}

\element{2016_kr_04_final}{ % в фигурных скобках название группы вопросов
%  %\AMCnoCompleteMulti
\begin{questionmult}{3} % тип вопроса (questionmult — множественный выбор) и в фигурных — номер вопроса
Обобщенный МНК служит для оценивания регрессионной модели $Y=X\beta + \e$ в случае нарушения следующего условия теоремы Гаусса-Маркова

\begin{multicols}{3} % располагаем ответы в 3 колонки
\begin{choices} % опция [o] не рандомизирует порядок ответов
       \correctchoice{$\Var(\e_i)=\sigma^2_{\e}$}
       \wrongchoice{$\E(\e_i)=0$}
       \wrongchoice{$\rank X = k$}
       \wrongchoice{$\Cov(\e_i, X_i)=0$}
       \wrongchoice{$\Cov(Y_i, \e_i)=0$}
    \end{choices}
\end{multicols}
\end{questionmult}
}


\element{2016_kr_04_final}{ % в фигурных скобках название группы вопросов
%  %\AMCnoCompleteMulti
\begin{questionmult}{4} % тип вопроса (questionmult — множественный выбор) и в фигурных — номер вопроса
При работе с панельными данными для выбора между моделью с фиксированными эффектами и моделью со случайными эффектами используется
\begin{multicols}{2} % располагаем ответы в 3 колонки
\begin{choices} % опция [o] не рандомизирует порядок ответов
       \correctchoice{тест Хаусмана}
       \wrongchoice{тест Бройша–Пагана}
       \wrongchoice{тест Голдфелда-Квандта}
       \wrongchoice{тест отношения правдоподобия}
       \wrongchoice{поиск на сетке}
    \end{choices}
\end{multicols}
\end{questionmult}
}


\element{2016_kr_04_final}{ % в фигурных скобках название группы вопросов
%  %\AMCnoCompleteMulti
\begin{questionmult}{5} % тип вопроса (questionmult — множественный выбор) и в фигурных — номер вопроса
При отсутствии автокорреляции в регрессии по $n$ наблюдениям статистика Дарбина-Уотсона имеет
\begin{multicols}{3} % располагаем ответы в 3 колонки
\begin{choices} % опция [o] не рандомизирует порядок ответов
       \wrongchoice{$t_n$-распределение}
       \wrongchoice{$\cN(0;1)$-распределение}
       \wrongchoice{$\cN(\mu;\sigma^2)$-распределение}
       \correctchoice{$F_{k,n}$-распределение}
       \wrongchoice{$t_{n-k}$-распределение}
    \end{choices}
\end{multicols}
\end{questionmult}
}



\element{2016_kr_04_final}{ % в фигурных скобках название группы вопросов
%  %\AMCnoCompleteMulti
\begin{questionmult}{6} % тип вопроса (questionmult — множественный выбор) и в фигурных — номер вопроса
При оценивании коэффициентов моделей бинарного выбора
\begin{multicols}{2} % располагаем ответы в 3 колонки
\begin{choices} % опция [o] не рандомизирует порядок ответов
       \wrongchoice{оценки логит и пробит моделей всегда совпадают}
       \wrongchoice{оценки логит моделей всегда выше, чем пробит}
       \wrongchoice{оценки пробит моделей всегда выше, чем логит}
       \wrongchoice{оценки логит и пробит моделей имеют противоположные знаки}
       \wrongchoice{оценки пробит модели имеют более высокую значимосить, чем логит}
    \end{choices}
\end{multicols}
\end{questionmult}
}


\element{2016_kr_04_final}{ % в фигурных скобках название группы вопросов
%  %\AMCnoCompleteMulti
\begin{questionmult}{7} % тип вопроса (questionmult — множественный выбор) и в фигурных — номер вопроса
Рассмотрим логит-модель $\hat Y_i^* = \hb_1 + \hb_2 X_i + \hb_3 D_i$, и $Y_i = 1$, если $Y_i^* > 0$. Если переменная $X_i$ является количественной, то предельный эффект увеличения $X_i$ можно посчитать по формуле
\begin{multicols}{3} % располагаем ответы в 3 колонки
\begin{choices} % опция [o] не рандомизирует порядок ответов
       \correctchoice{$\hb_2 f(\hat Y_i^*)$}
       \wrongchoice{$\hb_2 /f(\hat Y_i^*)$}
       \wrongchoice{$\hb_2 / F(\hat Y_i^*)$}
       \wrongchoice{$\hb_2 / f^2(\hat Y_i^*)$}
       \wrongchoice{$\hb_2 /F^2(\hat Y_i^*)$}
    \end{choices}
   \end{multicols}
\end{questionmult}
}

\element{2016_kr_04_final}{ % в фигурных скобках название группы вопросов
%  %\AMCnoCompleteMulti
\begin{questionmult}{8} % тип вопроса (questionmult — множественный выбор) и в фигурных — номер вопроса
Процесс $\e_t$ является белым шумом. Нестационарным является процесс
\begin{multicols}{3} % располагаем ответы в 3 колонки
\begin{choices} % опция [o] не рандомизирует порядок ответов
       \correctchoice{$Y_t = -1Y_{t-1} + \e_t$}
       \wrongchoice{$Y_t = 2017\e_t$}
       \wrongchoice{$Y_t = 0.1Y_{t-1} + \e_t$}
       \wrongchoice{$Y_t = 2017$}
       \wrongchoice{$Y_t = \e_t + 0.1\e_{t-1} + 1.5\e_{t-2}$}
    \end{choices}
   \end{multicols}
\end{questionmult}
}

\element{2016_kr_04_final}{ % в фигурных скобках название группы вопросов
%  %\AMCnoCompleteMulti
\begin{questionmult}{9} % тип вопроса (questionmult — множественный выбор) и в фигурных — номер вопроса
Использование робастных стандартных ошибок в форме Уайта при гетероскедастичности позволяет
\begin{multicols}{2} % располагаем ответы в 3 колонки
\begin{choices} % опция [o] не рандомизирует порядок ответов
       \correctchoice{строить корректные доверительные интервалы для коэффициентов}
       \wrongchoice{получить эффективные оценки коэффициентов}
       \wrongchoice{устранить смещённость оценок коэффициентов}
       \wrongchoice{увеличить точность прогнозов}
       \wrongchoice{сузить доверительные интервалы для коэффициентов}
    \end{choices}
   \end{multicols}
\end{questionmult}
}

\element{2016_kr_04_final}{ % в фигурных скобках название группы вопросов
%  %\AMCnoCompleteMulti
\begin{questionmult}{10} % тип вопроса (questionmult — множественный выбор) и в фигурных — номер вопроса
Оценка регрессионной зависимости с помощью МНК по 1234 наблюдениям имеет вид $\hat Y_i = 1 - 3X_i + 4Z_i$. Оценка ковариационной матрицы имеет вид
\[
\Var(\hb)=
\begin{pmatrix}
1 & 0.1 & 0.2 \\
0.1 & 4 & 1.5 \\
0.2 & 1.5 & 18 \\
\end{pmatrix}.
\]

Длина 95\%-го доверительного интервала для $\beta_2 + \beta_3$ примерно равна

\begin{multicols}{3} % располагаем ответы в 3 колонки
\begin{choices} % опция [o] не рандомизирует порядок ответов
       \correctchoice{$20$}
       \wrongchoice{$10$}
       \wrongchoice{$25$}
       \wrongchoice{$5$}
       \wrongchoice{$1.96$}
    \end{choices}
   \end{multicols}
\end{questionmult}
}
