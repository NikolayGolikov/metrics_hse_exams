\element{2017_fall_retake_1}{ % в фигурных скобках название группы вопросов
%  %\AMCnoCompleteMulti
\begin{questionmult}{1} % тип вопроса (questionmult — множественный выбор) и в фигурных — номер вопроса

Для набора панельных данных истинна спецификация модели со случайными эффектами, однако Вовочка оценивает модель с фиксированными эффектами. Вовочкины ценки коэффициентов $\beta$ окажутся

\begin{multicols}{2} % располагаем ответы в 3 колонки
\begin{choices} % опция [o] не рандомизирует порядок ответов
       \correctchoice{состоятельными и неэффективными}
       \wrongchoice{несостоятельными}
       \wrongchoice{состоятельными и эффективными}
       \wrongchoice{смещёнными и неэффективными}
       \wrongchoice{несмещёнными и эффективными}
    \end{choices}
   \end{multicols}
\end{questionmult}
}


\element{2017_fall_retake_1}{ % в фигурных скобках название группы вопросов
%  %\AMCnoCompleteMulti
\begin{questionmult}{2} % тип вопроса (questionmult — множественный выбор) и в фигурных — номер вопроса

Винни-Пух пытается понять, от каких переменных может зависеть его потребление мёда. Собрав 100 разных переменных, он построил 100 парных регрессий и проверил в них значимость коэффициента при каждой из переменных на уровне значимости 0.05. Пятачок понимает, что все 100 переменных не имеют никакого отношения к потреблению мёда и на самом деле просто случайные числа. Помогите Пятачку определить, сколько значимых переменных скорее всего найдёт Винни-Пух.

\begin{multicols}{3} % располагаем ответы в 3 колонки
\begin{choices} % опция [o] не рандомизирует порядок ответов
       \correctchoice{5}
       \wrongchoice{10}
       \wrongchoice{0}
       \wrongchoice{100}
       \wrongchoice{Не хватает данных для ответа}
    \end{choices}
   \end{multicols}
\end{questionmult}
}


\element{2017_fall_retake_1}{ % в фигурных скобках название группы вопросов
%  %\AMCnoCompleteMulti
\begin{questionmult}{3} % тип вопроса (questionmult — множественный выбор) и в фигурных — номер вопроса

Общеизвестно, что потребление мёда Винни-Пухом зависит, при этом положительно, от количества стихов, сочинённых им за день. К сожалению, Винни-Пух забывчив и всегда называет число сочинённых им стихов с ошибкой. Тогда оценка $\beta_1$ в регрессии $Honey_i = \beta_0 + \beta_1 Poems_i + \varepsilon_i $ окажется

%\begin{multicols}{3} % располагаем ответы в 3 колонки
\begin{choices}[o] % опция [o] не рандомизирует порядок ответов
       \correctchoice{Несостоятельной, заниженной}
       \wrongchoice{Несостоятельной, завышенной}
       \wrongchoice{Несостоятельной}
       \wrongchoice{Смещённой, но состоятельной}
       \wrongchoice{Несмещенной, но не состоятельной}
    \end{choices}
   %\end{multicols}
\end{questionmult}
}


% bad style
\element{2017_fall_retake_1}{ % в фигурных скобках название группы вопросов
%  %\AMCnoCompleteMulti
\begin{questionmult}{4} % тип вопроса (questionmult — множественный выбор) и в фигурных — номер вопроса

Из откровений внеземного разума известно, что эндогенности в модели $Y_i = \beta_0 + \beta_1 X_i + \varepsilon_i$ нет. Однако Вовочка нашёл хороший инструмент $z_i$, отвечающий всем требованиям, предъявляемым к инструментам, и оценил $\beta_1$ методом инструментальных переменных. Его оценка $\beta_1$ окажется

\begin{multicols}{3} % располагаем ответы в 3 колонки
\begin{choices} % опция [o] не рандомизирует порядок ответов
       \correctchoice{состоятельной, но не эффективной}
       \wrongchoice{состоятельной и эффективной}
       \wrongchoice{несостоятельной}
       \wrongchoice{состоятельной}
       \wrongchoice{невозможно сказать по имеющимся данным}
    \end{choices}
   \end{multicols}
\end{questionmult}
}





\element{2017_fall_retake_1}{ % в фигурных скобках название группы вопросов
%  %\AMCnoCompleteMulti
\begin{questionmult}{5} % тип вопроса (questionmult — множественный выбор) и в фигурных — номер вопроса

Рассмотрим процесс $Y_t = -0.2 Y_{t-1} + \varepsilon_t$. 5-ое значение автокорреляционной функции равно

\begin{multicols}{1} % располагаем ответы в 3 колонки
\begin{choices} % опция [o] не рандомизирует порядок ответов
       \correctchoice{ -0.00032 }
       \wrongchoice{ 0.00032 }
       \wrongchoice{ 0.2 }
       \wrongchoice{ -0.2 }
       \wrongchoice{ 0 }
    \end{choices}
   \end{multicols}
\end{questionmult}
}



\element{2017_fall_retake_1}{ % в фигурных скобках название группы вопросов
%  %\AMCnoCompleteMulti
\begin{questionmult}{6} % тип вопроса (questionmult — множественный выбор) и в фигурных — номер вопроса

Модель коррекции ошибками имеет следующий вид


\begin{multicols}{2} % располагаем ответы в 3 колонки
\begin{choices} % опция [o] не рандомизирует порядок ответов
       \correctchoice{$\Delta Y_t = \delta + \phi \Delta X_{t-1} - \gamma (Y_{t-1} - \alpha - \beta X_{t-1}) + \varepsilon_t$}
       \wrongchoice{$Y_t = \delta + \phi \Delta X_{t-1} - \gamma (Y_{t-1} - \alpha - \beta X_{t-1}) + \varepsilon_t$}
       \wrongchoice{$\Delta Y_t = \delta - \gamma (Y_{t-1} - \alpha - \beta X_{t-1}) + \varepsilon_t$}
       \wrongchoice{$Y_t = \delta - \gamma (Y_{t-1} - \alpha - \beta X_{t-1}) + \varepsilon_t$}
       \wrongchoice{$\Delta Y_t = \delta + \phi \Delta X_{t-1} - \gamma (Y_{t-1} ) + \varepsilon_t$}
    \end{choices}
   \end{multicols}
\end{questionmult}
}







\element{2017_fall_retake_1}{ % в фигурных скобках название группы вопросов
% \AMCnoCompleteMulti
\begin{questionmult}{7} % тип вопроса (questionmult — множественный выбор) и в фигурных — номер вопроса

Пусть $\varepsilon_t$ - белый шум. Тогда стационарным будет следующий процесс

\begin{multicols}{3} % располагаем ответы в 3 колонки
\begin{choices} % опция [o] не рандомизирует порядок ответов
       \correctchoice{$Y_t = \sum_{i = 0}^{10} \varepsilon_{t-i}$}
       \wrongchoice{$Y_t = 2018t + \varepsilon_t$}
       \wrongchoice{$Y_t = t \varepsilon_t$}
       \wrongchoice{$Y_t = Y_{t-1} - \varepsilon_t$}
       \wrongchoice{$Y_t = 2Y_{t-1} - \varepsilon_t$}
    \end{choices}
   \end{multicols}
\end{questionmult}
}





\element{2017_fall_retake_1}{ % в фигурных скобках название группы вопросов
%  %\AMCnoCompleteMulti
\begin{questionmult}{8} % тип вопроса (questionmult — множественный выбор) и в фигурных — номер вопроса

Процесс случайного блуждания с дрейфом описывается уравнением

\begin{multicols}{2} % располагаем ответы в 3 колонки
\begin{choices} % опция [o] не рандомизирует порядок ответов
       \correctchoice{$X_t = \mu + X_{t-1} + \varepsilon_t$}
       \wrongchoice{$X_t = \mu + 0.7 X_{t-1} + \varepsilon_t$}
       \wrongchoice{$X_t = X_{t-1} + \varepsilon_t$}
       \wrongchoice{$X_t = 0.7 X_{t-1} + \varepsilon_t$}
       \wrongchoice{$X_t = \mu + \varepsilon_t$}
    \end{choices}
   \end{multicols}
\end{questionmult}
}







\element{2017_fall_retake_1}{ % в фигурных скобках название группы вопросов
 %\AMCnoCompleteMulti
\begin{questionmult}{9} % тип вопроса (questionmult — множественный выбор) и в фигурных — номер вопроса

Если процесс является стационарным в широком смысле, то

\begin{multicols}{3} % располагаем ответы в 3 колонки
\begin{choices} % опция [o] не рандомизирует порядок ответов

       \wrongchoice{ Он является стационарным в узком смысле }
       \wrongchoice{ Для него выполняется основная гипотеза в тесте Дикки-Фуллера }
       \wrongchoice{ Его автоковариационная функция является постоянной }
       \wrongchoice{ Это белый шум }
       \wrongchoice{ Это AR процесс с корнями характеристического уравнения, меньшими 1 }
    \end{choices}
   \end{multicols}
\end{questionmult}
}




\element{2017_fall_retake_1}{ % в фигурных скобках название группы вопросов
 %\AMCnoCompleteMulti
\begin{questionmult}{10} % тип вопроса (questionmult — множественный выбор) и в фигурных — номер вопроса

При оценивании регрессионной модели $Y_t = a_0 + \sum_{j=1}^3 a_j X_{jt} + \varepsilon_t$ по 20 наблюдениям получено значение статистики Дарбина-Уотсона $d = 0.8$. При уровне значимости 1\% это свидетельствует о


\begin{multicols}{3} % располагаем ответы в 3 колонки
\begin{choices} % опция [o] не рандомизирует порядок ответов
       \correctchoice{Попадании в зону неопределенности}
       \wrongchoice{Положительной автокорреляции}
       \wrongchoice{Отрицательной автокорреляции}
       \wrongchoice{Отсутствии автокорреляции}
       \wrongchoice{Тест Дарбина-Уотсона вообще не проверяет наличие автокорреляции}
%       \wrongchoice{3 и 4}
    \end{choices}
   \end{multicols}
\end{questionmult}
}
