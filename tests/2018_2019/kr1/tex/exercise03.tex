
\begin{question}
Чебурашка оценил модель \(Y_i = \beta_0 + \beta_1 X_i + \varepsilon_i\),
а Крокодил Гена --- модель \(X_i = \gamma_0 + \gamma_1 Y_i + u_i\).
Оказалось, что \(\hat\gamma_1 = 0.25/\hat\beta_1\). Величина \(R^2\) в
регрессии Чебурашки равна
\begin{answerlist}
  \item \(1\)
  \item \(0\)
  \item \(0.75\)
  \item \(0.5\)
  \item \(0.25\)
\end{answerlist}
\end{question}

\begin{solution}
\(R^2 = \hat\beta_1 \cdot \hat\gamma_1\)
\end{solution}

