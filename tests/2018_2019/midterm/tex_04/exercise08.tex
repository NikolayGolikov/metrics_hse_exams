
\begin{question}
(1 балл) Чудо-швабры производятся на разных заводах по одной из двух технологий,
\(A\) или \(B\). Исследователь оценил две модели зависимости выпуска,
\(Y\), от количества сырья, \(X\), и технологии:

\(\hat Y_i = \hat \alpha_0 + \hat\alpha_1 A_i + \hat\alpha_2 X_i + \hat\alpha_3 A_i X_i\);

\(\hat Y_i = \hat \beta_0 + \hat\beta_1 B_i + \hat\beta_2 X_i + \hat\beta_3 B_i X_i\).

Переменная \(A_i\) равна единице для заводов с технологией \(A\) и нулю
иначе, а переменная \(B_i\) равна единице для заводов с технологией
\(B\) и нулю иначе.

Оценки коэффициентов связаны соотношением
\begin{answerlist}
  \item \(\hat\alpha_0 = \hat\beta_0\)
  \item \(\hat\alpha_0 + \hat\alpha_1 = \hat\beta_0\)
  \item \(\hat\alpha_1 = \hat\beta_0\)
  \item \(\hat\alpha_2 = \hat\beta_2\)
  \item \(\hat\alpha_0 = \hat\beta_0 + \hat\beta_1\)
  \item нет верного ответа
\end{answerlist}
\end{question}

\begin{solution}
\begin{answerlist}
  \item Bad answer :(
  \item Good answer :)
  \item Bad answer :(
  \item Bad answer :(
  \item Bad answer :(
\end{answerlist}
\end{solution}
